\documentclass[12pt]{report}

\usepackage[ngerman]{babel}
\usepackage{geometry}
\usepackage{mathtools}
\usepackage(pgfplots)
\geometry{
	a4paper,
	left=25mm,
	top=25mm,
	right=25mm,
	bottom=30mm
}

%opening
\title{Konstruktion eines 6-achsigen Roboterarms}

\author{
	Vladimir Morozov\\
	Gymnasium Bäumlihof\\
	Betreuer: Dominik Rohner\\
}

\begin{document}


\maketitle


\begin{abstract}
	Zusammenfassung
\end{abstract}
\clearpage

\tableofcontents
\cleardoublepage

\section{Vorwort}
\begin{equation}
R(\psi) =
\begin{bmatrix}
	\cos{\psi} & -\sin{\psi} & 0 \\
	\sin{\psi} & \cos{\psi} & 0 \\
	0 & 0 & 1 \\
\end{bmatrix};\\
R(\theta) =
\begin{bmatrix}
	\cos{\theta} & 0 & \sin{\theta} \\
	0 & 1 & 0 \\
	-\sin{\theta} & 0 & \cos{\theta} \\
\end{bmatrix};
R(\phi) =
\begin{bmatrix}
	1 & 0 & 0 \\
	0 & \cos{\theta} & -\sin{\theta} \\ 0 & \sin{\theta} & \cos{\theta} \\
\end{bmatrix};
\end{equation}
\end{document}
